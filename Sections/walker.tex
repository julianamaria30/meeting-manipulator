%*----------- SLIDE -------------------------------------------------------------
\begin{frame}[c]{} 
    % \transdissolve[duration=0.5]
   
    \begin{center}
        \Wider{%
        \begin{shaded}
        \begin{center}
            \vspace*{0.3cm}
            \resizebox{!}{0.6cm}{%
                Walker
            }%
        \end{center}
        \end{shaded}
        }%
    \end{center}
       
%*----------- notes
    \note[item]{Notes can help you to remember important information. Turn on the notes option.}
\end{frame}
%
%*----------- SLIDE -------------------------------------------------------------
\begin{frame}[t]{Introdução} 
    Este projeto consiste em desenvolver um robô de pequeno porte que se desloca sobre dois pés. O robô deve ser capaz de se locomover e desviar de obstáculos em um determinado ambiente. Além disso, os \textbf{objetivos especifícos} são:
   
        \begin{columns}[c]
            \column{.6\textwidth}
                \begin{itemize}
                    \item Desenvolver algoritmos utilizando ROS;
                    \item Utilizar visão computacional;
                    \item Simular um robô no gazebo;
                    \item Desenvolver habilidades de gestão de projetos.
                \end{itemize}
            \column{.25\textwidth}
                \includegraphics[width=1.1\textwidth]{walker3D}
        \end{columns}

\end{frame}
%-

%*----------- SLIDE -------------------------------------------------------------
\begin{frame}[t]{Missão do robô} 
    O Walker deve realizar uma missão, em um ambiente real, que consiste em:
    \newline
    \begin{columns}[c]
        \column{.6\textwidth}
            \begin{itemize}
                \item Navegar por um ambiente, de 
                forma autônoma;
                \item Reconhecer uma TAG;
                \item Encontrar um objeto (esfera) 
                em determinada região.
            \end{itemize}
        \column{.35\textwidth}
                \hspace{2cm}\includegraphics[width=0.3
                \textwidth]{aruco}
                \newline
                \includegraphics[width=0.3
                \textwidth]{esfera}
    \end{columns}
        
\end{frame}
%-


%*----------- SLIDE -------------------------------------------------------------
\begin{frame}[t]{Requisitos do Cliente} 
    \begin{columns}[c]
        \column{.6\textwidth}
            \begin{itemize}
                \item Operar em uma área de 2m x 1,5m;
                \item Possuir uma altura de aproximadamente 30 cm;
                \item Ser capaz de operar por, no mínimo, 20 min;
                \item Ser capaz de desviar de obstáculos.
            \end{itemize}
        \column{.35\textwidth}
            \includegraphics[width=0.7\textwidth]{biped}
    \end{columns}

\end{frame}
%-


%*----------- SLIDE -------------------------------------------------------------
\begin{frame}[t]{Requisitos técnicos} 
\begin{columns}
    \column{.0\textwidth}
    \column{.30\textwidth}
        \includegraphics[width=\textwidth]{walker_3Dfrontal}
    \column{.7\textwidth}
        \begin{enumerate}
            \item Uso de servomotores com torque adequado ao deslocamento de uma caraga de massa 2kg
            \item Uso de sensor ativo para detecção de objetos a uma distância de 2,5m 
            \item Uso de câmera de espectro vísivel 
            \item ARM com 4GB de RAM
            \item ROS
            \item Ubuntu 20.04 + ROS Noetic
            \item Baterias Lithium de 10V a 14,8V
            \item Massa máxima de 2kg
        \end{enumerate}
\end{columns}

\end{frame}
%-


%*----------- SLIDE -------------------------------------------------------------
\begin{frame}[t]{Estrutura Analítica do Projeto} 
    \centering
    \includegraphics[width=\textwidth]{EAP}
        
\end{frame}
%-


%*----------- SLIDE -------------------------------------------------------------
\begin{frame}[t]{Cronograma} 
    \begin{columns}
        \column{.57\textwidth}
            \includegraphics[width=1.1\textwidth]{cronograma_geral}
        \column{.41\textwidth}
            \begin{itemize}
                \item \textbf{Conceptual:} Semanas 0 a 4
                \item \textbf{Design:} Semanas 4 a 6
                \item \textbf{Development:} Semanas 6 a 10
                \item \textbf{Conclusion:} Semanas 10 a 11
            \end{itemize}
    \end{columns}
        
\end{frame}
%-


%*----------- SLIDE -------------------------------------------------------------
\begin{frame}[t]{Arquitetura Geral do Sistema} 

    \centering
    \includegraphics[width=.8\textwidth]{arquitetura_geral}

\end{frame}
%-


%*----------- SLIDE -------------------------------------------------------------
\begin{frame}[t]{Pesquisa por similares}
\begin{columns}
    \column{.35\textwidth}
        \centering    
            \newline \newline 
            \hspace{1.5cm}\includegraphics[width=.55\textwidth]{biped.png}
                \\ ROFI 
    \column{.35\textwidth}
        \centering
            \includegraphics[width=.65\textwidth]{darwin.png}
            \\ Darwin-OP 
    \column{.35\textwidth}
        \centering
            \includegraphics[width=.65\textwidth]{nao.png}
            \\ NAO
\end{columns}
\end{frame}
%-


%*----------- SLIDE -------------------------------------------------------------
\begin{frame}[t]{Product Breakdown Structure (PBS)} 
\centering
\vspace{1cm}
\includegraphics[width=\textwidth]{PBS_walker.png}
\end{frame}
%-


%*----------- SLIDE -------------------------------------------------------------
\begin{frame}[t]{Elaboração do QFD} 

    Com o objetivo de incorporar as necessidades do cliente ao projeto,
    \\ foi realizado o CFD (Desdobramento da Função Qualidade)
    \begin{columns}
        \column{.4\textwidth}
            \footnotesize
            \begin{itemize}
                \item \textbf{Técnicos x Técnicos} 
                \begin{itemize}
                    \footnotesize
                    \item Fortemente positivo  $\bullet$
                    \item Positivo $\circ$
                    \item Negativo $\ast$
                    \item Fortemente negativo $\diamondsuit$
                \end{itemize}
                \footnotesize
                \item \textbf{Cliente x Técnicos}
                \begin{itemize}
                    \footnotesize
                    \item Relações fortes $\bullet $
                    \item Relações médias $\circ $
                    \item Relações fracas $\bigtriangleup$
                \end{itemize}
            \end{itemize}
        \column{.56\textwidth}
            \vspace{-1.55cm}
            \includegraphics[width=\textwidth]{QFD}
    \end{columns}

\end{frame}


%*----------- SLIDE -------------------------------------------------------------
\begin{frame}[t]{Especificação de Funcionalidades} 
    \centering
    \includegraphics[width=.68\textwidth]{funcionalidades}

\end{frame}
%-


\begin{frame}[t]{Andamento do projeto} 
    \begin{columns}
        \column{.6\textwidth} 
            \centering \textbf{Realizado X Planejado: } x\%
            \begin{table}[]
                \begin{tabular}{@{}
                >{\columncolor[HTML]{C8D2EC}}l 
                >{\columncolor[HTML]{C8D2EC}}c @{}}
                \toprule
                \multicolumn{1}{c}{\cellcolor[HTML]{C8D2EC}{\color[HTML]{333333} \textbf{Biped Robot}}} & {\color[HTML]{333333} \textbf{36\%$\rightarrow$55\% }} \\ \midrule
                {\color[HTML]{333333} \textbf{Conceptual}}                                                       & {\color[HTML]{333333} \textbf{100\%$\rightarrow$100\%}} \\
                \textbf{Design}                                                                                  & \textbf{61\%$\rightarrow$96\%}                        \\
                \textbf{Development}                                                                             & \textbf{0\%$\rightarrow$9\%}                         \\
                {\color[HTML]{333333} \hspace{.2cm} Definition}                                                  & 0\%$\rightarrow$9\%                  \\
                \hspace{.2cm} Integration                                                                        & 0\%                                  \\
                \hspace{.2cm} Tests in Simulation                                                                & 0\%                                  \\
                \hspace{.2cm} Construction                                                                       & 0\%                                  \\
                \hspace{.2cm} Physical tests                                                                     & 0\%                                  \\
                \textbf{Conclusion}                                                                              & \textbf{0\%}                         \\ \bottomrule
                \end{tabular}
            \end{table}
        \column{.5\textwidth} 
            \includegraphics[width=1\textwidth]{andamento}
    \end{columns}
\end{frame}


%*----------- SLIDE -------------------------------------------------------------
\begin{frame}[t]{2 ciclo do QFD} 
    \begin{columns}
        \column{.4\textwidth}
        \footnotesize
        % \begin{itemize}
            \footnotesize
            % \item \textbf{Cliente x Técnicos}
            % \begin{itemize}
                %     \footnotesize
                %     \item Relações fortes $\bullet $
                %     \item Relações médias $\circ $
                %     \item Relações fracas $\bigtriangleup$
                % \end{itemize}
                \textbf{Targets} 
                \begin{itemize}
                    \footnotesize
                    \item Realizar a comunicação e a aquisição dos dados
                    \item Realizar a odometria e a atualização da posição
                    \item Realizar o mapeamento da área
                    \item Definir e planejar a trajetória
                    \item Elaborar os movimentos do roô
                    \item Definir a missão e o modo de operação
                    \item Avaliar as situações de alimentação e comportamento
                    \item Realizar a comunicação e aquisição de dados com os atuadores
                \end{itemize}
                % \end{itemize}
        \column{.4\textwidth}
            \newline
            \vspace{-0.55cm}
            \includegraphics[width=1.2\textwidth]{2qfd}
    \end{columns}
    
\end{frame}
%-
%*----------- SLIDE -------------------------------------------------------------
\begin{frame}[t]{Shield PCB} 
    \begin{columns}
        \column{.5\textwidth}
        \includegraphics[width=\textwidth]{shield}

        \column{.5\textwidth}
        \includegraphics[width=1.1\textwidth]{shield_onshape.png}
        \vspace{-1cm}

    \end{columns}
\end{frame}
%-


%*----------- SLIDE -------------------------------------------------------------
\begin{frame}[t]{Desenhos Mecânicos} 
    \begin{columns}
        \column{.22\textwidth}
        \includegraphics[width=\textwidth]{walker3D}

        \column{.22\textwidth}
        \includegraphics[width=\textwidth]{walker_3Dfrontal}

    \end{columns}
\end{frame}
%-


%*----------- SLIDE -------------------------------------------------------------
\begin{frame}[t]{Montagem do URDF} 
    \begin{columns}
        \column{.65\textwidth}
        \centering
        \includegraphics[width=.92\textwidth]{walker_urdf}

        \column{.35\textwidth}
        \textbf{Criação dos pacotes:}
        \begin{itemize}
            \item walker\_description
            \item walker\_gazebo
        \end{itemize}
        \vspace{.5cm}
        \textbf{URDF:}
        \begin{itemize}
            \item Criação e posicionamento dos links
            \item Adição dos plugins dos sensores
            \item Transmissão dos motores
        \end{itemize}
    \end{columns}
\end{frame}
%-

%*----------- SLIDE -------------------------------------------------------------
\begin{frame}[t]{Initial Design} 

    \centering
    \includegraphics[width=.9\textwidth]{initial_design.png}

\end{frame}
%-

%*----------- SLIDE -------------------------------------------------------------
\begin{frame}[t]{Análise de riscos do projeto} 
    \begin{enumerate}
        \item Incompatibilidade da comunicação entre dynamixels e ROS Noetic
        \begin{enumerate}
            \item Adaptar os pacotes disponíveis para o ROS Melodic para o Noetic
        \end{enumerate}   
        \item Falta de informações sobre o mapa coletadas pelo sensor ultrassônico 
        \begin{enumerate}
            \item Incluir mais dois sensores, cada um na lateral do robô
            \item Substituir sensor ultrassônico por um LIDAR
        \end{enumerate}  
    \end{enumerate}
    
\end{frame}
%-