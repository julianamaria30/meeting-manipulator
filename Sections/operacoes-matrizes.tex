%*----------- SLIDE -------------------------------------------------------------
\begin{frame}[t]{Operações básicas com matrizes} 
    \framesubtitle{Soma e subtração}
 
    As \textbf{matrizes} devem ter a mesma dimensão. \\
    \vspace*{0.3cm}
    Operação de soma
    \begin{equation}
        \begin{pmatrix}
            1 & 2 \\
            3 & 4 \\
            5 & 6 \\
        \end{pmatrix}
        +
        \begin{pmatrix}
            6  & 5 \\
            4  & 3 \\
            2  & 1 \\
        \end{pmatrix}
        =
        \begin{pmatrix}
            1 + 6  & 2 + 5  \\
            3 + 4  & 4 + 3  \\
            5 + 2  & 6 + 1  \\
        \end{pmatrix}
        =
        \begin{pmatrix}
            7  & 7 \\
            7  & 7 \\
            7  & 7 
        \end{pmatrix}
    \end{equation}
    \vspace*{0.3cm}
    Operação de subtração
    \begin{equation}
        \begin{pmatrix}
            1 & 2 \\
            3 & 4 \\
            5 & 6 \\
        \end{pmatrix}
        -
        \begin{pmatrix}
            6  & 5 \\
            4  & 3 \\
            2 & 1 \\
        \end{pmatrix}
        =
        \begin{pmatrix}
            1 - 6  & 2 - 5  \\
            3 - 4  & 4 - 3  \\
            5 - 2  & 6 - 1  \\
        \end{pmatrix}
        =
        \begin{pmatrix}
           -5  & -3 \\
           -1  &  1 \\
            3  &  5 \\
        \end{pmatrix}
    \end{equation}
%*----------- notes
    \note[item]{Notes can help you to remember important information. Turn on the notes option.}
\end{frame}
%-

%*----------- SLIDE -------------------------------------------------------------
\begin{frame}[t]{Operações básicas com matrizes} 
    \framesubtitle{Multiplicação escalar}

    Operação de multiplicação por escalar
    \begin{equation}
        10
        *
        \begin{pmatrix}
            1 & 2 \\
            3 & 4 \\
            5 & 6 \\
        \end{pmatrix}
        =
        \begin{pmatrix}
            10 * 1  & 10 * 2  \\
            10 * 3  & 10 * 4  \\
            10 * 5  & 10 * 6  \\
        \end{pmatrix}
        =
        \begin{pmatrix}
            10  & 20 \\
            30  & 40\\
            50  & 60 
        \end{pmatrix}
    \end{equation}
    \vspace*{0.3cm}
%*----------- notes
    \note[item]{Notes can help you to remember important information. Turn on the notes option.}
\end{frame}
%-
%*----------- SLIDE -------------------------------------------------------------
\begin{frame}[t]{Operações básicas com matrizes} 
    \framesubtitle{Multiplicação}

    As \textbf{matrizes} só podem ser multiplicadas somente se o número de colunas no fator da esquerda é igual ao número de linhas no fator da direita. \\
    \begin{equation}
        m x \textcircled{n} * \textcircled{n} x p = m x p
    \end{equation}
    \vspace*{0.3cm}
    Operação de multiplicação entre matrizes
    \begin{equation}
        \begin{pmatrix}
            1 & 2 \\
            3 & 4 \\
            5 & 6 \\
        \end{pmatrix}
        *
        \begin{pmatrix}
            x_1 & y_1 \\
            x_2 & y_2 \\
        \end{pmatrix}
        =
        \begin{pmatrix}
            1x_1 + 2x_2  & 1y_1 + 2y_2  \\
            3x_1 + 4x_2  & 3y_1 + 4y_2  \\
            5x_1 + 6x_2  & 5y_1 + 6y_2  \\
        \end{pmatrix}
    \end{equation}
    \vspace*{0.3cm}
%*----------- notes
    \note[item]{Notes can help you to remember important information. Turn on the notes option.}
\end{frame}
%-