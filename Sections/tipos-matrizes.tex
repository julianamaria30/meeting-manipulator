%*----------- SLIDE -------------------------------------------------------------
\begin{frame}[t]{Tipos de matrizes} 
    \framesubtitle{Matriz zero}
 
    A \textbf{matriz zero} possui todos os elementos iguais a zero.
    \vspace*{0.8cm}

    \begin{columns}[c]
        \column{.3\textwidth}
        \begin{equation}
            \begin{pmatrix}
                0 & 0 \\
                0 & 0 \\
            \end{pmatrix}
        \end{equation}
    
       \column{.35\textwidth}
       \begin{equation}
        \begin{pmatrix}
            0 \\
            0 \\
            0 \\
            0
            \end{pmatrix}
        \end{equation}

        \column{.35\textwidth}
        \begin{equation}
         \begin{pmatrix}
             0 & 0  & 0\\
             0 & 0  & 0\\

             \end{pmatrix}
         \end{equation}

   \end{columns}
    
%*----------- notes
    \note[item]{Notes can help you to remember important information. Turn on the notes option.}
\end{frame}
%-

%*----------- SLIDE -------------------------------------------------------------
\begin{frame}[t]{Tipos de matrizes} 
    \framesubtitle{Matriz transposta}

    A \textbf{matriz transposta} troca as linhas e as colunas em uma matriz.
    \begin{equation}
        \begin{bmatrix}
            1 & 3 & 5\\
            2 & 4 & 6\\
         \end{bmatrix}^\top =
            \begin{bmatrix}
            1 & 2 \\
            3 & 4 \\
            5 & 6   
         \end{bmatrix}
    \end{equation}
    \vspace*{0.3cm}
%*----------- notes
    \note[item]{Notes can help you to remember important information. Turn on the notes option.}
\end{frame}
%-
%*----------- SLIDE -------------------------------------------------------------
\begin{frame}[t]{Tipos de matrizes} 
    \framesubtitle{Matriz simétrica}

    A \textbf{matriz simétrica} é uma matriz que é simétrica em volta da sua diagonal principal. Por causa desta característica, a matriz simétrica é sempre igual a sua transposta. 
    \begin{equation}
            \begin{bmatrix}
            \textcolor{red}{1} & 5 & 6 & 7\\
            5 & \textcolor{red}{2} & 8 & 9\\
            6 & 8 & \textcolor{red}{3} & 10\\ 
            7 & 9 & 10 & \textcolor{red}{4}\\ 
         \end{bmatrix}
    \end{equation}
    \vspace*{0.3cm}
%*----------- notes
    \note[item]{Notes can help you to remember important information. Turn on the notes option.}
\end{frame}
%-
%*----------- SLIDE -------------------------------------------------------------
\begin{frame}[t]{Tipos de matrizes} 
    \framesubtitle{Matriz triangular superior e triangular inferior}

    A \textbf{matriz triangular} é uma matriz quadrada onde os elementos ou acima ou abaixo da diagonal principal são todos iguais a zero.
    \vspace*{0.8cm}


    \begin{columns}[c]
        \column{.5\textwidth}
        \centering
        Matriz triangular superior
        \begin{equation}
                \begin{bmatrix}
                \textcolor{red}{1} & 5 & 6 & 7\\
                0 & \textcolor{red}{2} & 8 & 9\\
                0 & 0 & \textcolor{red}{3} & 10\\ 
                0 & 0 & 0 & \textcolor{red}{4}\\ 
             \end{bmatrix}
        \end{equation}
    
       \column{.5\textwidth}
       \centering
       Matriz triangular inferior
       \begin{equation}
               \begin{bmatrix}
               \textcolor{red}{1} & 0 & 0 & 0\\
               5 & \textcolor{red}{2} & 0 & 0\\
               6 & 8 & \textcolor{red}{3} & 0\\ 
               7 & 9 & 10 & \textcolor{red}{4}\\ 
            \end{bmatrix}
       \end{equation}
   \end{columns}
%*----------- notes
    \note[item]{Notes can help you to remember important information. Turn on the notes option.}
\end{frame}
%-
%*----------- SLIDE -------------------------------------------------------------
\begin{frame}[t]{Tipos de matrizes} 
    \framesubtitle{Matriz diagonal}

    A \textbf{matriz diagonal} é uma matriz quadrada onde todos os elementos que não fazem parte da diagonal principal são iguais a zero.
    \vspace*{0.5cm}

    \begin{equation}
            \begin{bmatrix}
            \textcolor{red}{1} & 0 & 0 & 0\\
            0 & \textcolor{red}{2} & 0 & 0\\
            0 & 0 & \textcolor{red}{3} & 0\\ 
            0 & 0 & 0 & \textcolor{red}{4}\\ 
         \end{bmatrix}
    \end{equation}
%*----------- notes
    \note[item]{Notes can help you to remember important information. Turn on the notes option.}
\end{frame}
%-

%*----------- SLIDE -------------------------------------------------------------
\begin{frame}[t]{Tipos de matrizes} 
    \framesubtitle{Matriz identidade}

    A \textbf{matriz identidade} é uma matriz quadrada com \textit{n} linhas onde todos os elementos na diagonal principal são iguais a 1 e todos os outros elementos são 0.
    \vspace*{0.5cm}

    \begin{equation}
            \begin{bmatrix}
            \textcolor{red}{1} & 0 & 0 & 0\\
            0 & \textcolor{red}{1} & 0 & 0\\
            0 & 0 & \textcolor{red}{1} & 0\\ 
            0 & 0 & 0 & \textcolor{red}{1}\\ 
         \end{bmatrix}
    \end{equation}
%*----------- notes
    \note[item]{Notes can help you to remember important information. Turn on the notes option.}
\end{frame}
%-
%*----------- SLIDE -------------------------------------------------------------
\begin{frame}[t]{Tipos de matrizes} 
    \framesubtitle{Matriz inversa}

    Se o produto de duas \textbf{matrizes quadradas} é uma \textbf{matriz identidade}, então as duas matrizes são \textbf{inversas} uma da outra.


    \begin{equation}
        \begin{pmatrix}
            1 & 2 \\
            3 & 4 \\
        \end{pmatrix}
        *
        \begin{pmatrix}
            x_{11} & x_{12} \\
            x_{21} & x_{22} \\
        \end{pmatrix}
        =
        \begin{pmatrix}
            1 & 0 \\
            0 & 1 \\
        \end{pmatrix}
    \end{equation}
%*----------- notes
    \note[item]{Notes can help you to remember important information. Turn on the notes option.}
\end{frame}
%-